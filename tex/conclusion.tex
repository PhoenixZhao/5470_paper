\section{CONCLUSION AND FUTURE WORK}
Many researchers have tried to integrate the social or trust relations between users to the matrix factorization framework to improve the recommendation performance in recent years. However, different from their work, my work is focused on the condition that such relations are not accessible directly, which is based on an assumption that people tend to trust those whose share similar preferences with them. Apart from this, observing that implicit structure exits between items, I also argue that people trust different people regarding different domains, which is very similar to \cite{yang2012circle}. Finally, the IDSR is proposed and a gradient descent method is employed to solve this matrix factorization problem with implicit domain-specific trust regularization. Experimental analysis shows the effectiveness of the proposed model.

Speaking of future work, one direction is that all the domains are considered i.i.d in this work, while they may have some hierarchical structure, which is a very intuitive idea from the observation that movies at Netflix are classified into a hierarchical structure as genre $\rightarrow$ sub-genre $\rightarrow$ detailed-category. Therefore, we should consider this hierarchical structure of all the domains when modeling this problem. However, it will increase hugely the computational complexity, which can make it very costly to calculate the result. This work is will be a future work of my research.